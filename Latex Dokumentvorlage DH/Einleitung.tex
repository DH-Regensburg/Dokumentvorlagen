\section{Einleitung}
Zu Beginn einer Seminararbeit, Masterarbeit oder eines Projektbericht steht 
immer die Einleitung. Eine Einleitung beträgt in etwa 5-10 \% der Gesamtanzahl der Seiten und beinhaltet die folgenden Aspekte:
\begin{itemize}
    \item \textbf{Ausblick auf die nachfolgende Arbeit}: Im ersten Absatz der Einleitung sollten die wichtigsten Punkte der Arbeit angesprochen und die Aufmerksamkeit der Lesenden erregt werden. Die Arbeit kann z.B. durch weltaktuelle Themen, relevante Fakten oder aktuelle Ereignisse begründet werden. 
    \item  \textbf{Themenpräsentation:} Im Rahmen der Themenpräsentation wird die wissenschaftliche Relevanz des Themas zu einem gewissen Forschungsgebiet der Digital Humanities begründet. Außerdem werden hier Schlüsselbegriffe erklärt. Es muss hier dargelegt werden, inwiefern die gewonnenen Erkenntnisse bedeutsam sind. Am Ende der Themenrepräsentation steht die spezifische Forschungsfrage der Arbeit.
    \item  \textbf{Ziel der Arbeit:} Hier werden Thesen aufgestellt, also wahrscheinliche Ergebnisse der Arbeit. Diese werden im Verlauf der Arbeit einer Validierung (oder Falsifizierung) unterzogen. In der Einleitung können darum noch keine Forschungsergebnisse oder eigenen Interpretationen 
    \item \textbf{Eingesetzte Methodik:} Um zu verstehen, wie es zur Klärung der Problemstellung gekommen ist, muss die Forschungsmethodik aufgezeigt werden. Dies umfasst Experimente, Interviews, Umfragen, Machine Learning vs. Deep Learning, Quantitativ vs. Qualitativ etc.stehen.
    \item  \textbf{Aufbau der Arbeit:} Für eine verbesserte Leserführung kann hier auch  schon der Aufbau der Arbeit skizziert werden.
\end{itemize}