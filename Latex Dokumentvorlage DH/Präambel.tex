\documentclass[12pt,a4paper]{article}
\usepackage[utf8]{inputenc}
\usepackage[backend=biber, style=authoryear, natbib=true]{biblatex}
\addbibresource{literatur.bib} 
\usepackage[T1]{fontenc}
\usepackage[german]{babel}
\usepackage{graphicx}
\usepackage{newtxtext,newtxmath} % Serifenschrift
%\usepackage[sfdefault]{FiraSans} % Entkommentieren für serifenlose Schrift
\usepackage{geometry}
\usepackage{hyperref}
\usepackage{titlesec}
\usepackage{tocloft}
\usepackage{setspace}
\usepackage{fancyhdr}
\usepackage{enumitem}


%Kopzeile/Fußzeile anpassen
\pagestyle{fancy}
\fancyhf{} %Löschen der gesetzten Formatierung
%Kopfzeile anpassen
\renewcommand{\headrulewidth}{0.4pt}
\renewcommand{\sectionmark}[1]{\markboth{#1}{}}
\fancyhead[C]{\footnotesize\nouppercase\leftmark}
%Fußzeile anpassen
\fancyfoot[C]{\thepage}

% Setzen der Seitenränder
\geometry{
  left=3.7cm,
  right=3.5cm,
  top=2.5cm,
  bottom=2.5cm
}

% Anpassung der Überschriften, um sie fett und in entsprechender Größe zu setzen
\titleformat{\section}
  {\Large\bfseries}{\thesection}{1em}{}
\titleformat{\subsection}
  {\large\bfseries}{\thesubsection}{1em}{}
\titleformat{\subsubsection}
  {\normalsize\bfseries}{\thesubsubsection}{1em}{}

% Fußnotengröße anpassen
\renewcommand{\footnotesize}{\small} % Für 10pt im 12pt Dokument

% Einrückung und Zeilenabstand
\setlength{\parindent}{0.7cm}
\onehalfspacing

%langes Zitat
\usepackage{etoolbox} % Für \newtoggle und \toggletrue/\togglefalse
\usepackage{setspace} % Für Zeilenabstandsänderungen
\newtoggle{langeszitat}
\togglefalse{langeszitat} % Standardmäßig ist das Zitat nicht aktiv
\newcommand{\langeszitat}[1]{%
  \toggletrue{langeszitat}% Schaltet um auf Langzitat-Modus
  \vspace{0.8em}
  \begin{quote}
  \vspace{-\topsep} % Entfernt zusätzlichen Abstand oben
  \leftskip=1cm \rightskip=1cm % Einrückung rechts und links
  \small % Eine Punktgröße kleiner als der Fließtext
  \setstretch{1.0} % Einstellung des Zeilenabstands auf 1.0
  #1
  \vspace{0.8em}
  \end{quote}
  \togglefalse{langeszitat}% Schaltet zurück nach dem Zitat
}

%Anpassen der Listenelemente, wenn Abstand nicht passend
\setlist[itemize]{itemsep=0pt, parsep=0pt} % Für ungeordnete Listen
\setlist[enumerate]{itemsep=0pt, parsep=0pt} % Für geordnete Listen
