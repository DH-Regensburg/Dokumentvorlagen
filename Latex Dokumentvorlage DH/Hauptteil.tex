\section{Hauptteil}
Der Hauptteil einer Master- oder Seminararbeit besteht meist aus den \textbf{theoretischen Grundlagen}, der \textbf{Beschreibung der genutzten Ressourcen}, \textbf{Tools und Methoden}, der \textbf{Dokumentation und Analyse der Experimente und Studien} und den daraus \textbf{resultierenden Ergebnissen}. Der Hauptteil nimmt etwa 80 \% der gesamten Arbeit ein. Der Hauptteil kann wie folgt untergliedert werden:
\begin{enumerate}
    \item \textbf{Aktueller Forschungstand:} Wie passt das Thema in den aktuellen Forschungsdiskurs?, welche unterschiedlichen Ansätze gibt es bereits?, welche technischen Aspekte wurden schon untersucht?. Hierbei geht es um eine systematische Übersicht, belegt durch Auswertung aktueller Forschungsergebnisse und Forschungsliteratur. Es gilt, Ergebnisse immer auf die eigene Forschung zu beziehen.
    \item \textbf{Methodik:} In diesem Abschnitt werden einerseits die genutzten Ressourcen und andererseits eingesetzte Tools und Methoden dokumentiert und reflektiert. Der Abschnitt weist die folgenden Unterpunkte auf:
    \begin{enumerate}
        \item \textit{Corpus:} Welche Quellen wurden wie und warum gesammelt? Wie wurden die Daten aufbereitet? Nach welchen Kriterien ist das Korpus aufgebaut?
        \item \textit{Methoden:} Welches ist die Hauptmethode, die eingesetzt wird oder wie ist die Methodenpipeline aufgebaut? Welche Tools werden genutzt? 
        \item \textit{Methodenkritik:} Warum wurde die eingesetzte Methodik gewählt und welche Alternativen wären möglich gewesen? Was kann mit der gewählten Methode untersucht werden? Was kann damit nicht untersucht werden? Welche eventuellen Fallstricke gilt es zu bedenken und wie begegnen Sie diesen in ihrer Arbeit? 
    \end{enumerate}
    \item \textbf{Dokumentation der Studie:} Welche Ergebnisse wurden generiert? Wie interpretieren Sie diese? Inwiefern können Sie Ihre Ergebnisse mit anderen Erkenntnissen des Forschungsfeldes verknüpfen?
    
\end{enumerate}