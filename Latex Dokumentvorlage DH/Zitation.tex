\section{Zitation}
Übernommene fremde Inhalte jeglicher Form und Herkunft (das gilt auch für Forschungssoftware, Forschungsdaten, Notebooks anderer Forschender u.Ä.) müssen über Referenzen bzw. Zitate angegeben werden. Grundsätzlich wird nach \textbf{Harvard} zitiert.
\subsection{Belege im Fließtext}
Im Fließtext werden Belege immer nach der zu belegenden Stelle in Klammern angeben (vgl. \cite [S.~50]{Musterfrau2015}). Direkte Zitate sind wortgenau zu übernehmen, „Fehler“ im Zitat können mit [sic] gekennzeichnet werden. Ausgebesserte Fehler und Auslassungen werden durch [] gekennzeichnet. „Bei direkten Zitaten wird die Abkürzung vgl. weggelassen“ (\cite [S.~65]{Mustermann2023}). Maxi  \textcite [S.~7]{Musterkind2019} sagt, dass die Seitenanzahl des Beleges immer nach dem Doppelpunkt folgt. Werden Autor*innen namentlich im Text genannt, so wird bei der Erstnennung Vor- und Nachname genutzt. Bei zwei Autor*innen werden diese zusammen angeben (vgl. \cite [S. ~2]{Herzog/König}), ab drei Autor*innen werden diese abgekürzt (vgl. \cite [S.~77] {Maier/Müller/Bauer/Fischer}. Mehrere Belege werden mit einem Semikolon getrennt (vgl. \cite [S. ~9] {Frühling}; \cite [S. ~28]{Sommer}). 
Sollte eine Autorin in einem Jahr mehrere Werke veröffentlich haben, so sind diese mit Kleinbuchstaben zu kennzeichnen, z.B. \cite{Maier1} \cite{Maier2}.\newline
Längere Zitate ab ca. 40 Wörter werden um mindestens 1 cm recht und links ein-gerückt. Die Schriftgröße wird um 1pt verkleinert und der Zeilenabstand auf 1,0 verringert. Sie werden ohne Anführungsstriche dargestellt:
\langeszitat{Habe nun, ach! Philosophie, Juristerei und Medizin, Und leider auch Theologie Durchaus studiert, mit heißem Bemühn [sic]. Da steh ich nun, ich armer Tor! Und bin so klug als wie zuvor; Heiße Magister, heiße Doktor gar Und ziehe schon an die zehen [sic] Jahr Herauf, herab und quer und krumm Meine Schüler an der Nase herum – Und sehe, da[ss]wir nichts wissen können!  (von Goethe, 2012)}
\subsection{Literaturverzeichnis}
Alle im Text genannten Referenzen müssen in alphabetischer Reihenfolge im Literaturverzeichnis gelistet werden. Ob Vornamen abgekürzt oder ausgeschrieben werden, kann nach eigenem Empfinden entschieden werden. Achten Sie dabei aber auf Einheitlichkeit!\\
Nutzen Sie folgende Form im Literaturvezeichnis:
\begin{itemize}
    \item Monografie:\\Nachname, Vorname (Jahr): \textit{Titel}, Stadt: Verlag.
    \item Fachzeitschrift:\\Nachname, Vorname (Jahr): “Artikeltitel”, in: \textit{Zeitschriftentitel}, Bd., Nr., Seitenzahl.
    \item Internetquelle (keine DOI vorhanden)\\Nachname, Vorname (Jahr): “Beitragstitel”, \textit{Website}, [online] URL [abgerufen am TT.MM.JJ].
    \item Internetquelle (DOI vorhanden)\\ Nachname, Vorname (Jahr): “Beitragstitel”, \textit{Website}, DOI: URL.
    \item 2 Autor*innen:\\ Nachname, Vorname und Nachname, Vorname
    \item Ab 3 Autor*innen:\\Nachname, Vorname / Nachname, Vorname und Nachname, Vorname

    \end{itemize}