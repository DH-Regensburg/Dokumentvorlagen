\noindent\textbf{Zusammenfassung}\\
\noindent
Masterarbeiten und Projektberichte beginnen meist mit einem Abstract bzw. einer Zusammenfassung. Diese dienen dazu, die wichtigsten Inhalte der vorliegenden Arbeit wiederzugeben und so die Leser*innen neugierig zu machen. Es 
gilt, einen Überblick über die Arbeit, Zielsetzung, Methodik, Gliederungspunkte und Ergebnisse zu geben. Die Länge eines Abstracts beträgt maximal eine 
Seite (ca. 200 Wörter). Die Zusammenfassung befindet sich noch vor der eigentlichen Arbeit, also direkt nach dem Deckblatt

\bigskip
\noindent\textbf{Abstract}\\
Es ist gern gesehen, dass das Abstract auch in englischer Sprache zu Verfügung  gestellt wird. Gemäß \href{https://www.uni-regensburg.de/assets/studium/pruefungsordnungen/magister-master/digital-humanities-20200713.pdf}{Prüfungsordnung} ist es den Studierenden frei gestellt, ob sie ihre Arbeit im Deutschen oder im Englischen verfassen. Sollte die Arbeit auf Englisch verfasst und ein Abstract vorhanden sein, so ist ein deutsches Abstract ebenso bereitzustellen.
\thispagestyle{empty}
