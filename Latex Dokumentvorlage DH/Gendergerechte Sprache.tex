\section{Gendergerechte Sprache}
Gendern ist weder verpflichtend, noch verboten.
Sollte Sie sich für eine gendergerechte Sprache entscheiden, so beachten Sie den \href{https://www.uni-regensburg.de/assets/rechtsgrundlagen/leitfaden-gendergerechte-sprache.pdf}{Leitfaden zur Verwendung gendergerechter Sprache der Universität Regensburg}. Bei der Umformung von Wörtern können folgende Schreibweisen verwendet werden: Binnen-I, Gender-Stern, Unterstrich, Doppelpunkt. Achten Sie dabei bitte auf Einheitlichkeit!
