\section{Gestaltungsrichtlinien}
\subsection{Sprache und Textumfang}
Laut Prüfungsordnung kann die Arbeit in deutscher oder englischer Sprache verfasst werden. Für andere Aufgabenformate gilt die Absprache mit den jeweiligen Prüfer*innen. Bei dem Umfang einzelner Aufgabentypen muss man sich an folgende Vorgaben halten:
\begin{itemize}
    \item Hausarbeit 30.000-40.000 Zeichen
    \item Short Paper 6.000-8.000 Zeichen
    \item Abstract 1.500-2.000 Wörter; hier bitte andere Vorlagen (z.B. acl, jcl) verwenden
    \item Masterarbeit \textbf{maximal} 80 Seiten
\end{itemize}
\subsection{Formatierung}
Die Arbeit wird typischerweise einseitig verfasst und im DIN A4- Format gedruckt. Die Formatierung ist wie folgt:
\begin{itemize}
    \item Oben: 2,5 cm
    \item Unten: 2,5 cm
    \item Links: 3,7 cm
    \item Rechts: 3,5 cm
\end{itemize}
Die Seiten werden am unteren Seitenrand rechts mit einer Seitenzahl versehen. Deckblatt und Inhaltsverzeichnis werden mitgezählt, aber nicht mitnummeriert. Die Nummerierung ist auf der ersten Seite der Einleitung zum ersten Mal zu sehen. Wenn gewollt, kann in der Kopfzeile für den jeweiligen Abschnitt entsprechend die Überschrift 1 eingefügt werden.

Alle Abschlussarbeiten sind in angemessener Weise (Klemm- oder Klebebindung für M.A.-Arbeiten) zu binden. Für Projektberichte und einfache Seminar-arbeiten genügt in Absprache mit den Prüfer*innen meist ein elektronisches Format (pdf). Wenn Sie Ihre Arbeit in gedruckter Form abgeben, nutzen Sie für die Bindung bitte ein Schnellhefter o.Ä., wenn gewollt auch eine Ringbindung. 
\subsubsection{Formatvorlage}
Für Schriftarten und Formatierung können die hier im Dokument vorgenommenen Einstellungen übernommen werden. Es wird keine explizite Schriftart gewünscht, sie sollte dennoch dem Anlass angemessen sein. Standartschriftarten für Fließtext und Fußnoten sind ‚Times New Roman‘, ‚Palatino‘, ‚Calibri‘ oder ‚Helvetica‘. Für die Überschriften können, müssen aber nicht die gleichen Schriftarten genommen werden. Wenn Sie Ihre Arbeit elektronisch einreichen, können Sie auch serifenlose Schriften wählen wie z.B. Arial.

Als Schriftgröße sollten für den Fließtext 12pt bei Serifenschriften und 11pt bei serifenlosen Schriften gewählt werden, Überschrift 3pt größer für weitere Untergliederungen entsprechende Größen wählen. Fußnoten sollten kleiner sein als der Fließtext (9 bzw. 10pt). Der Fließtext ist als Blocksatz und linksbündig zu schreiben. Die erste Zeile eines Absatzes wird um 0,7 cm eingerückt.
\subsubsection{Inhaltliche Bestandteile}
Fast jede Arbeit besteht aus folgenden Bestandteile, in der vorgegebenen Reihenfolge. Es ist darauf zu achten, dass jeder Abschnitt auf einer neuen Seite beginnt.
\begin{enumerate}
    \item Deckblatt
    \item Abstract
    \item Inhaltsverzeichnis
    \item Einleitung
    \item Hauptteil
    \item Schluss
    \item Optional: Projektbeteiligung
    \item Literaturverzeichnis
    \item Abbildungsverzeichnis
    \item Tabellenverzeichnis
    \item Plagiatserklärung
    \item Anhang
\end{enumerate}
Das Deckblatt kann aus dieser Datei übernommen und mit den entsprechenden fehlenden Informationen gefüllt werden (nicht genutzte Punkte können gelöscht werden). Es fehlen Angaben zum Semester, Veranstaltungstitel, Dozent*in, Thementitel, sowie alle Angaben zum Autor bzw. zur Autorin, diese müssen durch eigene Angaben ergänzt werden. 

Bei Verzeichnissen sollten Sie vermeiden, dass mehr als vier Untergliederungspunkte genutzt werden, gewünscht sind etwa drei Unterpunkte. Verzeichnisse und Plagiatserklärung werden im Inhaltsverzeichnis aufgeführt, aber nicht mitnummeriert.

\subsubsection{Abbildungen, Tabellen, Code}
Abbildungen und Tabellen sind jeweils mit Titel zu versehen. Abbildungen fordern darüber hinaus auch noch eine Quellenangabe. Sollten diese keinen Fremdinhalt beinhalten, so können sie als ‚eigenes Bild‘ gekennzeichnet werden. Bei Abbildungen sollte ebenso auf eine passende Auflösung für die Lesbarkeit geachtet werden ggf. Schicken Sie eine hochauflösende Datei (svg) mit oder legen sie in einem Forschungsdatenrepositorium ab (z.B. auf GitHub). Sollten Abbildungen und Tabellen genutzt werden, benötigen sie ein eigenes Verzeichnis.

Die Darstellung von Code erfolgt als Text und nicht über das Einfügen von Abbildungen. Die Schriftart sollte sich vom restlichen Text abheben (z.B. ‚Courier New‘). Das Einfügen von Codeabschnitten darf den Lesefluss nicht stark unterbrechen und sollte deshalb 20 Zeilen nicht überschreiten. Möchten Sie ganze Notebooks als Ergänzung Ihrer Arbeit einreichen, so legen Sie diese bitte in einem Forschungsdatenrepositorium (z.B. auf GitHub) ab und verlinken Sie dieses als Referenz im Text. 

Betten Sie Abbildungen, Tabellen und Code bitte durch Verweise in Ihren Fließtext ein und achten Sie darauf, dass sie an der richtigen Stelle erscheinen.